\documentclass{article}
\usepackage{tikz}
\usepackage{CJKutf8}
\usepackage{amsmath}
\usepackage{amsthm}
\newtheorem{Ex}{练习}

\newtheorem*{Example}{例}

\begin{document}
\begin{CJK}{UTF8}{gbsn}
\underline{课程学习目标:}
\begin{enumerate}
\item 提升逻辑思维能力
  \begin{enumerate}
  \item 直接证明法
  \item 数学归纳法
  \item 反证法

    ...
  \end{enumerate}
\item 提升抽象思维能力
\item 提升利用数学语言准确描述计算机科学问题和电子信息科学问题的能力
\end{enumerate}

\underline{一个直接证明法的例子:}

证明:设$x, y, z$为任意三个非负有理数,则$(x + y) + z = x + (y + z)$。
\begin{proof}[证明]
  设$x=\frac{a}{b}$,$y=\frac{c}{d}$,$z=\frac{e}{f}$,其中$a$,$c$,$e$为自然数,$b$,$d$,$f$为正整数,
  则
  \begin{align*}
    (x + y) + z&=(\frac{a}{b}+\frac{c}{d})+\frac{e}{f}\\
               &=\frac{ad+bc}{bd}+\frac{e}{f}\\
    &=\frac{adf+bcf+bde}{bdf}\\
    x + (y + z)&=\frac{a}{b}+(\frac{c}{d}+\frac{e}{f})\\
               &=\frac{a}{b} + \frac{cf + de}{df}\\
    &=\frac{adf+bcf+bde}{bdf}    
  \end{align*}
  这证明了$(x + y) + z = x + (y + z)$。
\end{proof}

\underline{一个数学归纳法的例子:}

证明:设$a, b, c$为任意三个自然数,则$(a + b) + c = a + (b + c)$。

\begin{proof}[证明]
  用数学归纳法证明。

  1.当$c=0$时,$(a+b)+0=a+(b+0)$显然成立。

  2.对于任意的自然数$k$,假设当$c=k$时结论成立,往证当$c=k+1$时结论也成立。$(a+b)+(k+1)=((a+b)+k)+1$,而$a+(b+(k+1))=a+((b+k)+1)=(a+(b+k))+1$。由归纳假设,$(a+b)+k=a+(b+k)$,从而$(a+b)+(k+1)=a+(b+(k+1))$。
  结论得证。
\end{proof}


刻画自然数的Peano公理(1889年提出):
\begin{enumerate}
    \item 0为自然数。
    \item 如果$n$为自然数,那么$n++$也为自然数。
    \item 如果$n$为自然数,那么$n++ \neq 0$。
    \item 设$n$,$m$为自然数,如果$n \neq m$,那么$n++ \neq m++$。
    \item 如果$P(0)$成立,并且对于任意的自然数$n$,当$P(n)$成立时,$P(n++)$也成立,那么对任意的自然数$n$,$P(n)$成立。 
    \end{enumerate}

    在以上Peano公理的基础上,递归的定义自然数的加法如下:
    
    对任意的自然数$m$,
    \begin{enumerate}
    \item $m + 0 = m$
    \item 对任意的自然数$n$,$m + (n++) = (m + n)++$
    \end{enumerate}
    证明:设$a, b, c$为任意三个自然数,则$(a + b) + c = a + (b + c)$。

\begin{proof}[证明]
  用数学归纳法证明。

  1.当$c=0$时,$(a+b)+0=a+(b+0)$显然成立。

  2.对于任意的自然数$k$,假设当$c=k$时结论成立,往证当$c=k++$时结论也成立。$(a+b)+(k++)=((a+b)+k)++$,而$a+(b+(k++))=a+((b+k)++)=(a+(b+k))++$。由归纳假设,$(a+b)+k=a+(b+k)$,从而$(a+b)+(k++)=a+(b+(k++))$,  结论得证。
\end{proof}

\underline{一个反证法的例子:}

证明:$\sqrt{2}$是无理数。
\begin{proof}[证明]
  用反证法。假设$\sqrt{2}$为有理数,则存在互素的整数$m$和$n$使得$\sqrt{2}=\frac{n}{m}$,于是
  \begin{equation}\label{eq_1}
    2m^2=n^2
  \end{equation}
  这说明$n$为偶数,从而存在整数$k$,使得$n=2k$。带入\eqref{eq_1}得,$m^2=2k^2$,从而$m$为偶数,这与$m$和$n$互素矛盾,结论得证。
\end{proof}
  命题:可以判断真假的陈述句。通常,我们用$T$表示真,用$F$表示假。
  \begin{Example}\quad
    
    \begin{enumerate}
    \item 对任意的自然数$a,b,c$,$(a + b) + c = a + (b + c)$。(真命题)
    \item $\sqrt{2}$是无理数。(真命题)
    \item $\sqrt{2}$是有理数。(假命题)
    \item 设$f:[a,b] \to R$为一个Riemann可积函数,$F:[a,b] \to R$在$[a,b]$上满足$F'(x)=f(x)$,那么$\int_{a}^{b}f(x)dx = F(b) - F(a)$。(真命题)
    \item “集合”在数学上是一个不可定义的概念。(真命题)
    \item 任何一幅地图都可以用四种颜色进行着色,使得相邻的区域着以不同的颜色。
    \item 这句话是假的。(不能判断真假的陈述句,不是命题)
    \end{enumerate}
  \end{Example}
  谓词:命题的谓语部分。
  
  \begin{Example}\quad
    
    \begin{description}
    \item     [$P(x): x$ 是偶数] 这里$P$为一元谓词,表示“是偶数”。当$x$为某个确定的数字时,$P(x)$则对应一个命题。例如$P(2)$为真命题,$P(1)$为假命题。这里,$P$之所以被称为一元谓词,是因为$P(x)$只包含一个变量$x$。
    \item     [$P(x,y): x >y$]  这里$P$为二元谓词,表示$>$。当$x$和$y$为确定的数字时,$P(x,y)$则对应一个命题。例如$1>0$为真命题,$0>1$为假命题。这里,$P$之所以被称为二元谓词,是因为$P(x,y)$包含两个变量$x$和$y$。
    \end{description}
相应的,有三元谓词,四元谓词,......
\end{Example}

我们还可以用如下方式由谓词得到命题:

\begin{description}
\item [$\forall x P(x)$:] 对任意的$x$,$P(x)$。For All中的$A$上下颠倒可以得到$\forall$。
\item [$\exists x P(x)$:] 存在$x$,$P(x)$。There Exists中的$E$左右颠倒可以得到$\exists$。
\end{description}

命题可以由联结词$\lnot$,$\land$,$\lor$,$\to$,$\leftrightarrow$联结而构成复合命题。

设$p$为命题,则$\lnot p$表示“$p$不成立”。

 \begin{tabular}{c|c}
    p& $\lnot$ p\\
    \hline
    T&F\\
    F&T\\
  \end{tabular}

  设$p$和$q$为两个命题,则$p\land q$表示“$p$成立,并且$q$成立”。
  
  \begin{tabular}{cc|c}
    p& q& p $\land$ q\\
    \hline
    T&T&T\\
    T&F&F\\
    F&T&F\\
    F&F&F\\
  \end{tabular}

  设$p$和$q$为两个命题,则$p\lor q$表示“$p$成立,或者$q$成立”。
  
  \begin{tabular}{cc|c}
    p& q& p $\lor$ q\\
    \hline
    T&T&T\\
    T&F&T\\
    F&T&T\\
    F&F&F\\
  \end{tabular}

设$p$和$q$为两个命题,则$p\to q$表示“如果$p$成立,那么$q$成立”。  

    \begin{tabular}{cc|c}
    p& q& p $\to$ q\\
    \hline
    T&T&T\\
    T&F&F\\
    F&T&T\\
    F&F&T\\
    \end{tabular}\hspace{0.87cm}

    这里需要注意的是,当$p$为假时,则$p\to q$一定为真,这是所有数学家共同的约定。
    下面的例子可以帮助大家更好的理解其实我们已经用到了这个约定。

    对任意的实数$x$,当$x>1$时,$x^2 > 1$。该命题显然是真命题,可以符号化为$\forall x \; x > 1 \to x^2 > 1$。那么,既然对于任意的$x$,$x>1 \to x^2>1$成立,则

    1)当$x=2$时, $2 > 1 \to 2^2 >1$成立,这对应于以上真值表的第一行;

    2)当$x=0$时,$0 > 1 \to 0^2 > 1$成立,这对应于以上真值表的第四行;

    3)当$x=-2$时,$-2>1 \to (-2)^2 > 1$成立,这对应于以上真值表的第三行。
    
设$p$和$q$为两个命题,则$p\leftrightarrow q$表示“$p$等价于$q$”。  

  \begin{tabular}{cc|c}
    p& q& p $\leftrightarrow$ q\\
    \hline
    T&T&T\\
    T&F&F\\
    F&T&F\\
    F&F&T\\
  \end{tabular}

  请大家思考,设$p$,$q$,$r$为命题,则$(p\lor q)\land r$所代表的命题的含义是什么?$(p\land r)\lor (q \land r)$所代表的命题的含义是什么?这两个命题是等价的吗?
  我们可以通过枚举$p$,$q$,$r$依次取值为$T$和$F$时,$(p\lor q)\land r$和$(p\land r)\lor (q \land r)$同时取值为$T$或$F$,从而验证这两个命题是等价的,如下所示:

    \begin{tabular}{ccc|cc}
    $p$& $q$& $r$& $(p\lor q)\land r$&$(p\land r)\lor (q \land r)$\\
    \hline
    T&T&T&T&T\\
    T&F&T&T&T\\
    F&T&T&T&T\\
      F&F&T&F&F\\
    T&T&F&F&F\\
    T&F&F&F&F\\
    F&T&F&F&F\\
      F&F&F&F&F\\      
  \end{tabular}

  用同样的方法我们可以验证:

  $(p\land q)\lor r$与$(p\lor r)\land (q \lor r)$是等价的。

      \begin{tabular}{ccc|cc}
    $p$& $q$& $r$&$(p\land q)\lor r$ &$(p\lor r)\land (q \lor r)$\\
    \hline
    T&T&T&T&T\\
    T&F&T&T&T\\
    F&T&T&T&T\\
      F&F&T&T&T\\
    T&T&F&T&T\\
    T&F&F&F&F\\
    F&T&F&F&F\\
        F&F&F&F&F\\
  \end{tabular}


  $\lnot (p\land q)$与$\lnot p \lor \lnot q$是等价的。

      \begin{tabular}{cc|cc}
    $p$& $q$&$\lnot (p\land q)$ &$\lnot p \lor \lnot q$\\
    \hline
    T&T&F&F\\
    T&F&T&T\\
    F&T&T&T\\
      F&F&T&T\\      
  \end{tabular}


  $\lnot (p \lor q)$与$\lnot p \land \lnot q$是等价的。
  
      \begin{tabular}{cc|cc}
    $p$& $q$&$\lnot (p \lor q)$&$\lnot p \land \lnot q$\\
    \hline
    T&T&F&F\\
    T&F&F&F\\
    F&T&F&F\\
    F&F&T&T\\      
  \end{tabular}

  $p \to q$与$\lnot p \lor q$是等价的。
  
      \begin{tabular}{cc|cc}
    $p$& $q$&$p \to q$&$\lnot p \lor q$\\
    \hline
    T&T&T&T\\
    T&F&F&F\\
    F&T&T&T\\
    F&F&T&T\\      
  \end{tabular}
  

  我们还可以利用真值表检验$(p\to q) \land (p\to \lnot q) \to \lnot p$是永真的。
  
      \begin{tabular}{cc|cc}
    $p$& $q$&$(p\to q) \land (p\to \lnot q) \to \lnot p$\\
    \hline
    T&T&T\\
    T&F&T\\
    F&T&T\\
    F&F&T\\      
      \end{tabular}

      假设我约定"$\to$"的真值表如下:

    \begin{tabular}{cc|c}
    p& q& p $\to$ q\\
    \hline
    T&T&T\\
    T&F&F\\
    F&T&F\\
    F&F&T\\
    \end{tabular}\hspace{0.87cm}

    我们会发现复合命题$(p\to q) \land (p\to \lnot q) \to \lnot p$不是永真的,这将与我们关于“蕴含”的思维不相符。

    同时我们还会发现$p \to q$和$q\to p$在逻辑上是等价的。

    \begin{tabular}{cc|cc}
    p& q& p $\to$ q&q $\to$ p\\
    \hline
    T&T&T&T\\
    T&F&F&F\\
    F&T&F&F\\
    F&F&T&T\\
    \end{tabular}\hspace{0.87cm}

    这也与我们的思维习惯不相符。

    

  
  有些逻辑术语从外文翻译成中文时产生了不同的称谓,在本门课程中关于逻辑术语我们做如下的约定:


  The negation of a proposition $P$: $\lnot P$

  命题$P$的否定:$\lnot P$

  The converse of $P\to Q$: $Q \to P$

  命题$P\to Q$的逆命题:$Q\to P$



  The inverse of $P\to Q$: $\lnot P \to \lnot Q$
  
  在较深入的探讨数理逻辑的教材中,该概念用的很少,因此我们不给出具体的翻译称谓,在需要表达该概念时明确说明为$\lnot P \to \lnot Q$即可。

  
  The contrapositive of $P\to Q$: $\lnot Q \to \lnot P$

  命题$P\to Q$的逆否命题:$\lnot Q \to \lnot P$

  需要特别说明的是,命题$P\to Q$的否定为$\lnot (P \to Q)$,而不是$P \to \lnot Q$。这将和英文表达相一致:The negation of $P\to Q$ is $\lnot (P \to Q)$. 

  
  

  
\end{CJK}
\end{document}


%%% Local Variables:
%%% mode: latex
%%% TeX-master: t
%%% End:
