\documentclass{article}
\usepackage{tikz}
\usepackage{CJKutf8}
\usepackage{amsmath}
\usepackage{amsthm}
\begin{document}
\begin{CJK}{UTF8}{gbsn}
\newtheorem*{Exercise}{习题}
\begin{Exercise}
  毕业舞会上,小伙子与姑娘跳舞。已知每个小伙子至少与一个姑娘跳过舞,但未能与所有的姑娘跳过舞。同样的,每个姑娘也至少与一个小伙子跳过舞,但也未能与所有的小伙子跳过舞。证明:在所有参加舞会的小伙子与姑娘中,必可找到两个小伙子与两个姑娘,这两个小伙子中的每一个只与这两个姑娘中的一个跳过舞,而这两个姑娘中的每一个也只与这两个小伙子中的一个跳过舞。
\end{Exercise}
\begin{proof}[证明]设$b_1$为与姑娘跳舞最多的小伙子。由$b_1$未能与所有的姑娘跳过舞知,存在一个姑娘$g_2$,$b_1$未能与$g_2$跳过舞。由每个姑娘至少与一个小伙子跳过舞知,存在一个小伙子$b_2$与$g_2$跳过舞。在与小伙子$b_1$跳过舞的姑娘中,必存在一个姑娘$g_1$未能与小伙子$b_2$跳过舞,否则与$b_1$为与姑娘跳舞最多的小伙子矛盾。于是,$b_1$与$g_1$跳过舞,但未与$g_2$跳过舞;$b_2$与$g_2$跳过舞,但未与$g_1$跳过舞,结论得证。  
\end{proof}

\end{CJK}
\end{document}


%%% Local Variables:
%%% mode: latex
%%% TeX-master: t
%%% End:
