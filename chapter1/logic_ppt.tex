\documentclass{beamer}
\usepackage{ragged2e}
\usepackage{CJKutf8}
\usepackage{tikz}
\setbeamertemplate{theorems}[numbered]
\justifying\let\raggedright\justifying
\begin{document}
\begin{CJK*}{UTF8}{gbsn}

\newtheorem{Thm}{定理}[section]
\theoremstyle{definition}
\newtheorem{Def}{定义}[section]
\theoremstyle{example}
\newtheorem*{Ex}{例:}
\newtheorem{Exercise}{习题}

\date{}
\author{陈建文}

\title{逻辑初步}
\begin{frame}
  \titlepage
\end{frame}  
\section{课程学习目标}
\begin{frame}
  \frametitle{课程学习目标}
\begin{enumerate}
\item 提升逻辑思维能力
  \begin{enumerate}
  \item 直接证明法
  \item 数学归纳法
  \item 反证法

    ...
  \end{enumerate}
\item 提升抽象思维能力
\item 提升利用数学语言准确描述计算机科学问题和电子信息科学问题的能力
\end{enumerate}

\end{frame}
\begin{frame}
  \frametitle{Peano公理(1889年)}
  \begin{enumerate}
    \item 0为自然数。
    \item 如果$n$为自然数,那么$n++$也为自然数。
    \item 如果$n$为自然数,那么$n++ \neq 0$。
    \item 设$n$,$m$为自然数,如果$n \neq m$,那么$n++ \neq m++$。
    \item 如果$P(0)$成立,并且对于任意的自然数$n$,当$P(n)$成立时,$P(n++)$也成立,那么对任意的自然数$n$,$P(n)$成立。 
    \end{enumerate}  
  \end{frame}
  \begin{frame}
  \frametitle{自然数加法的定义}
        对任意的自然数$m$,
    \begin{enumerate}
    \item $m + 0 = m$
    \item 对任意的自然数$n$,$m + (n++) = (m + n)++$
    \end{enumerate}
  \end{frame}
  \begin{frame}
        证明:设$a, b, c$为任意三个自然数,则$(a + b) + c = a + (b + c)$。

\begin{proof}[证明]\justifying\let\raggedright\justifying

  用数学归纳法证明。

  1.当$c=0$时,$(a+b)+0=a+(b+0)$显然成立。

  2.对于任意的自然数$k$,假设当$c=k$时结论成立,往证当$c=k++$时结论也成立。$(a+b)+(k++)=((a+b)+k)++$,而$a+(b+(k++))=a+((b+k)++)=(a+(b+k))++$。由归纳假设,$(a+b)+k=a+(b+k)$,从而$(a+b)+(k++)=a+(b+(k++))$,
  结论得证。
\end{proof}
\end{frame}
\begin{frame}
  \frametitle{课程学习目标}
\begin{enumerate}
\item 提升逻辑思维能力
  \begin{enumerate}
  \item 直接证明法
  \item 数学归纳法
  \item 反证法

    ...
  \end{enumerate}
\item 提升抽象思维能力
\item 提升利用数学语言准确描述计算机科学问题和电子信息科学问题的能力
\end{enumerate}
\end{frame}
\begin{frame}
  "如果不理解它的语言,没有人能读懂宇宙这本伟大的书,它的语言就是数学"
  --伽利略(1564-1642)
\end{frame}
\begin{frame}
  \frametitle{课程学习目标}
\begin{enumerate}
\item 提升逻辑思维能力
  \begin{enumerate}
  \item 直接证明法
  \item 数学归纳法
  \item 反证法

    ...
  \end{enumerate}
\item 提升抽象思维能力
\item 提升利用数学语言准确描述计算机科学问题和电子信息科学问题的能力
\end{enumerate}
\end{frame}
\begin{frame}
  “如果$p$成立,那么$q$成立。”这句话的否定是()

  A. “如果$p$成立,那么$q$不成立。”

  B. “$p$成立并且$q$不成立。”
\end{frame}
\begin{frame}
    命题:可以判断真假的陈述句。通常,我们用$T$表示真,用$F$表示假。
  \begin{Ex}\quad   
    \begin{enumerate}
    \item 对任意的自然数$a,b,c$,$(a + b) + c = a + (b + c)$。(真命题)
    \item $\sqrt{2}$是无理数。(真命题)
    \item $\sqrt{2}$是有理数。(假命题)
    \item 设$f:[a,b] \to R$为一个Riemann可积函数,$F:[a,b] \to R$在$[a,b]$上满足$F'(x)=f(x)$,那么$\int_{a}^{b}f(x)dx = F(b) - F(a)$。(真命题)
    \item “集合”在数学上是一个不可定义的概念。(真命题)
    \item 任何一幅地图都可以用四种颜色进行着色,使得相邻的区域着以不同的颜色。
    \item 这句话是假的。(不能判断真假的陈述句,不是命题)
    \end{enumerate}
  \end{Ex}
\end{frame}
\begin{frame}
    谓词:命题的谓语部分。
  
  \begin{Ex}\quad
    
    \begin{description}
    \item     [$P(x): x$ 是偶数] 这里$P$为一元谓词,表示“是偶数”。当$x$为某个确定的数字时,$P(x)$则对应一个命题。例如$P(2)$为真命题,$P(1)$为假命题。这里,$P$之所以被称为一元谓词,是因为$P(x)$只包含一个变量$x$。
    \item     [$P(x,y): x >y$]  这里$P$为二元谓词,表示$>$。当$x$和$y$为确定的数字时,$P(x,y)$则对应一个命题。例如$1>0$为真命题,$0>1$为假命题。这里,$P$之所以被称为二元谓词,是因为$P(x,y)$包含两个变量$x$和$y$。
    \end{description}
相应的,有三元谓词,四元谓词,......
\end{Ex}
\end{frame}
\begin{frame}
  我们还可以用如下方式由谓词得到命题:

\begin{description}
\item [$\forall x P(x)$:] 对任意的$x$,$P(x)$。For All中的$A$上下颠倒可以得到$\forall$。
\item [$\exists x P(x)$:] 存在$x$,$P(x)$。There Exists中的$E$左右颠倒可以得到$\exists$。
\end{description}

\end{frame}
\begin{frame}
  \frametitle{复合命题}
  设$p$为命题,则$\lnot p$表示“$p$不成立”。

 \begin{tabular}{c|c}
    p& $\lnot$ p\\
    \hline
    T&F\\
    F&T\\
  \end{tabular}

\end{frame}
\begin{frame}
    \frametitle{复合命题}
    设$p$和$q$为两个命题,则$p\land q$表示“$p$成立,并且$q$成立”。
  
  \begin{tabular}{cc|c}
    p& q& p $\land$ q\\
    \hline
    T&T&T\\
    T&F&F\\
    F&T&F\\
    F&F&F\\
  \end{tabular}

\end{frame}
\begin{frame}
    \frametitle{复合命题}

    设$p$和$q$为两个命题,则$p\lor q$表示“$p$成立,或者$q$成立”。
  
  \begin{tabular}{cc|c}
    p& q& p $\lor$ q\\
    \hline
    T&T&T\\
    T&F&T\\
    F&T&T\\
    F&F&F\\
  \end{tabular}

\end{frame}
\begin{frame}
    \frametitle{复合命题}
  设$p$和$q$为两个命题,则$p\to q$表示“如果$p$成立,那么$q$成立”。  

    \begin{tabular}{cc|c}
    p& q& p $\to$ q\\
    \hline
    T&T&T\\
    T&F&F\\
    F&T&T\\
    F&F&T\\
    \end{tabular}
  \end{frame}
  \begin{frame}
      \frametitle{复合命题}
    设$p$和$q$为两个命题,则$p\leftrightarrow q$表示“$p$等价于$q$”。  

  \begin{tabular}{cc|c}
    p& q& p $\leftrightarrow$ q\\
    \hline
    T&T&T\\
    T&F&F\\
    F&T&F\\
    F&F&T\\
  \end{tabular}
\end{frame}

\begin{frame}
    \frametitle{一些逻辑术语的约定}
    The negation of a proposition $P$: $\lnot P$

  命题$P$的否定:$\lnot P$

  The converse of $P\to Q$: $Q \to P$

  命题$P\to Q$的逆命题:$Q\to P$



  The inverse of $P\to Q$: $\lnot P \to \lnot Q$
  
  在较深入的探讨数理逻辑的教材中,该概念用的很少,因此我们不给出具体的翻译称谓,在需要表达该概念时明确说明为$\lnot P \to \lnot Q$即可。

  
  The contrapositive of $P\to Q$: $\lnot Q \to \lnot P$

  命题$P\to Q$的逆否命题:$\lnot Q \to \lnot P$

  需要特别说明的是,命题$P\to Q$的否定为$\lnot (P \to Q)$,而不是$P \to \lnot Q$。这将和英文表达相一致:The negation of $P\to Q$ is $\lnot (P \to Q)$. 
\end{frame}
\end{CJK*}
\end{document}
