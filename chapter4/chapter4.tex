\documentclass{beamer}
%\usepackage{beamerthemesplit}
\usepackage{CJKutf8}
\usepackage{tikz}
\usepackage{clrscode3e}
\setbeamertemplate{theorems}[numbered]

\begin{document}
\begin{CJK*}{UTF8}{gbsn}

\newtheorem{Thm}{定理}[section]
\newtheorem{Cor}{推论}[section]
\newtheorem{Ax}{公理}[section]
\theoremstyle{definition}
\newtheorem{Def}{定义}[section]
\theoremstyle{example}
\newtheorem*{Ex}{例:}
\date{}
\author{陈建文}

\title{第四章 无穷集合及其基数}
\begin{frame}
  \titlepage
\end{frame}  
\section{可数集}
\begin{frame}
  \frametitle{1. 可数集}
  \begin{Def}
    如果从集合$X$到集合$Y$存在一个双射,则称$X$与$Y$\alert{对等},记为$X \sim Y$。
  \end{Def}\pause
  \begin{Def}
    如果从自然数集$\mathbb{N}$到集合$X$存在一个一一对应$f:\mathbb{N}\to X$,则称
    集合$X$为可数无穷集合,简称\alert{可数集}或\alert{可列集}。如果$X$不是可数集且$X$不是有穷集合,则称$X$为不可数无穷集合,简称\alert{不可数集}。
  \end{Def}\pause
  \begin{Thm}
    集合$A$为可数集的充分必要条件是$A$的全部元素可以排成无重复项的序列
    \[a_1, a_2, \ldots, a_n, \cdots\]
    因此,$A$可写成$A = \{a_1, a_2, a_3, \cdots\}$。
  \end{Thm}
\end{frame}

\begin{frame}
  \frametitle{1. 可数集}
  \begin{Thm}
   设$A$为可数集合,$B$为有穷集合,则$A\cup B$为可数集。
  \end{Thm}
\end{frame}

\begin{frame}
  \frametitle{1. 可数集}
  \begin{Thm}
    设$A$与$B$为两个可数集,则$A\cup B$为可数集。
  \end{Thm}
\end{frame}

\begin{frame}
  \frametitle{1. 可数集}
  \begin{Thm}
    设$A_1, A_2, \cdots, A_n, \cdots$为可数集合的一个无穷序列,则$\bigcup_{n=1}^{\infty}A_n$是可数集。即可数多个可数集之并是可数集。
  \end{Thm}
\end{frame}

\begin{frame}
  \frametitle{1. 可数集}
  \begin{Thm}
    设$A$与$B$为两个可数集,则$A\times B$为可数集。
  \end{Thm}
\end{frame}

\begin{frame}
  \frametitle{1. 可数集}
  \begin{Thm}
    全体有理数之集$\mathbb{Q}$是可数集。
  \end{Thm}
\end{frame}


\section{连续统集}
\begin{frame}
  \frametitle{2 连续统集}
  \begin{Thm}
    区间$[0,1]$中的所有实数构成的集合是不可数集。
  \end{Thm}
\end{frame}

\begin{frame}
  \frametitle{2 连续统集}
  \begin{Def}
    凡与集合$[0,1]$存在一个一一对应的集合称为具有“连续统的势”的集合,简称\alert{连续统}。
  \end{Def}
\end{frame}

\begin{frame}
  \frametitle{2 连续统集}
  \begin{Thm}
    无穷集合必包含有可数子集。
  \end{Thm}\pause
  \begin{Thm}
    设$M$是一个无穷集合,$A$是至多可数集合,则$M \sim M \cup A$。
  \end{Thm}\pause
  \begin{Thm}
    设$M$为无穷集合,$A$为$M$的至多可数子集,$M\setminus A$为无穷集合,则$M \sim M\setminus A$。
  \end{Thm}
\end{frame}
\begin{frame}
  \frametitle{2 连续统集}
  \begin{Thm}
    设$A_1, A_2, \cdots, A_n$为$n$个两两不相交的连续统,则$\bigcup_{i=1}^nA_i$是连续统。
  \end{Thm}
\end{frame}

\begin{frame}
  \frametitle{2 连续统集}
  \begin{Thm}
    设$A_1, A_2, \cdots, A_n, \cdots$为两两不相交的集序列。如果$A_k \sim [0,1], k = 1, 2, \cdots$,则
    \[\bigcup_{n=1}^{\infty}A_n \sim [0,1]\]
  \end{Thm}
  \begin{Cor}
    全体实数之集是一个连续统。
  \end{Cor}
  \begin{Cor}
    全体无理数之集是一个连续统。
  \end{Cor}
\end{frame}

\begin{frame}
  \frametitle{2 连续统集}
    \begin{codebox}
    \Procname{$\proc{K}(P)$}
    \li \If $H(P,P) == 1$ 
    \li  $\quad\quad$\Return
    \li \ElseNoIf Loop forever
    \End
  \end{codebox}  
\end{frame}

\section{基数及其比较}
\begin{frame}
  \frametitle{3 基数及其比较}
  \begin{Def}
    集合$A$的基数是一个符号,凡与$A$对等的集合都赋以同一个记号。集合$A$的基数记为$|A|$。
  \end{Def}
  \begin{Def}
    所有与集合$A$对等的集合构成的集族称为$A$的基数。
  \end{Def}
    \begin{Def}
    集合$A$的基数与集合$B$的基数称为是相等的,当且仅当$A \sim B$。
  \end{Def}
\end{frame}

\begin{frame}
  \frametitle{3 基数及其比较}
  \begin{Def}
    设$\alpha$,$\beta$为任意两个基数,$A$,$B$为分别以$\alpha$,$\beta$为其基数的
    集合。如果$A$与$B$的一个真子集对等,但$A$却不能与$B$对等,则称基数$\alpha$小于基数$\beta$,记为$\alpha < \beta$。
  \end{Def}\pause
  显然,

  $\alpha \leq \beta$当且仅当存在单射$f:A \to B$。

  $\alpha < \beta$当且仅当存在单射$f:A \to B$且不存在$A$到$B$的双射。
\end{frame}

\begin{frame}
  \frametitle{3 基数及其比较}
  \begin{Thm}[康托]
    对任一集合$M$,$|M| < |2^{M}|$。
  \end{Thm}
\end{frame}

\section{康托-伯恩斯坦定理}
\begin{frame}
  \frametitle{4 康托-伯恩斯坦定理}
  \begin{Thm}[康托-伯恩斯坦]
    设$A$,$B$为两个集合。如果存在单射$f:A\to B$与单射$g:B\to A$,则$A$与$B$的基数相等。
  \end{Thm}
\end{frame}


% \begin{frame}
%   \frametitle{4 康托-伯恩斯坦定理}
%   \begin{Thm}[塔斯基不动点定理]
%     设$(A, \leq)$为一个完备格($A$的任一非空子集均有上确界和下确界),$f:A \to A$为单调函数($\forall x, y \in A, x \leq y \rightarrow f(x) \leq f(y)$),则$\exists z \in A$,使得$f(z)=z$。
%   \end{Thm}
%   \begin{Thm}[巴拿赫映射分解定理]
%     设$A$, $B$为任意两个集合,$f$为从$A$到$B$的映射,$g$为从$B$到$A$的映射,则存在$A$的划分$\{A_1,A_2\}$和$B$的$\{B_1,B_2\}$,使得\[f(A_1) = B_1,g(A_2) = B_2.\]
%   \end{Thm}
% \end{frame}

\section{公理集合论}
\begin{frame}
  \frametitle{5 公理集合论}
  \begin{Ax}[外延公理]
    \begin{equation*}
      \forall A \forall B (\forall x (x \in A \leftrightarrow x\in B)\rightarrow A = B)
    \end{equation*}
  \end{Ax}
  \begin{Ax}[空集公理]
    \begin{equation*}
      \exists \phi \forall x (x \notin \phi)
    \end{equation*}
  \end{Ax}
  \begin{Ax}[对公理]
    \begin{equation*}
      \forall u \forall v \exists B \forall x (x \in B \Leftrightarrow x = u \lor x = v)
    \end{equation*}
  \end{Ax}
  \begin{Ax}[并集公理]
    \begin{equation*}
     \forall A \exists B \forall x (x \in B \leftrightarrow (\exists b \in A) x \in b)
    \end{equation*}
  \end{Ax}
\end{frame}
\begin{frame}
  \frametitle{5 公理集合论}
    \begin{Ax}[幂集公理]
    \begin{equation*}
      \forall a \exists B \forall x ( x \in B \leftrightarrow x \subseteq a)
    \end{equation*}
  \end{Ax}
  \begin{Ax}[子集公理]
    \begin{equation*}
      \forall c \exists B \forall x (x \in B \leftrightarrow x \in c \land \varphi(x))
    \end{equation*}
  \end{Ax}
  \begin{Ax}[无穷公理]
    \begin{equation*}
      \begin{split}
      \exists A ( \phi \in A \land (\forall a \in A) a^+ \in A)\\
      \text{其中} a^+ = a \cup \{a\}
      \end{split}
    \end{equation*}
  \end{Ax}
\end{frame}

\begin{frame}
  \frametitle{5 公理集合论}
    \begin{Ax}[代换公理]
    \begin{equation*}
      \begin{split}
      \forall A ((\forall x \in A) \forall y_1 \forall y_2 (\varphi(x, y_1) \land \varphi(x, y2) \rightarrow y_1 = y_2)\\
      \rightarrow \exists B \forall y (y \in B \leftrightarrow (\exists x \in A) \varphi(x, y)))
    \end{split}
  \end{equation*}
  \end{Ax}
  \begin{Ax}[正则公理]
    \begin{equation*}
      (\forall A \neq \phi) (\exists m \in A) m \cap A = \phi
    \end{equation*}
  \end{Ax}
  \begin{Ax}[选择公理]
    \begin{equation*}
      (\forall \text{relation} R)
      (\exists \text{function} F)
      (F \subseteq R \land
      \text{dom} F
      = \text{dom} R)
    \end{equation*}
  \end{Ax}
\end{frame}

\begin{frame}
  \frametitle{5 公理集合论}
  \begin{enumerate}
  \item $0 \in \mathbb{N}$;
  \item $n \in \mathbb{N} \rightarrow n ++ \in \mathbb{N}$;
  \item $\forall n \in \mathbb{N} n ++ \neq 0$;
  \item $\forall n \in \mathbb{N} \forall m \in \mathbb{N} n \neq m \rightarrow n ++ \neq m ++$;
    \item $(P(0) \land \forall n \in \mathbb{N} p(n) \rightarrow p(n++) )\rightarrow \forall n p(n)$。 
  \end{enumerate}
\end{frame}
\begin{frame}
  \frametitle{5 公理集合论}
   设$x, y, z \in \mathbb{R}$,则
   \begin{enumerate}
   \item   $x + y = y + x$
   \item   $(x + y) + z = x + (y + z)$
   \item   $0 + x = x + 0 = x$
   \item   $(-x) + x = x + (-x) = 0$
   \item   $x * y = y * x$
   \item   $(x * y) * z = x * (y *z)$
   \item   $1 * x = x * 1 = x$
   \item   $\forall x \in \mathbb{R} x \neq 0 \to x^{-1} * x = x * x^{-1} = 1$
   \item   $x* (y + z) = x * y + x * z$
   \item   $(y + z) * x = y * x + z * x$
   
    \end{enumerate}
  \end{frame}
  \begin{frame}
  \frametitle{5 公理集合论}
    \begin{enumerate}
       \item $x \leq x$
   \item $ x \leq y \land y \leq x \rightarrow x = y$
   \item $x \leq y \land y \leq z \rightarrow x \leq z$
   \item $x \leq y \lor y \leq x$ 
\item $x > y \rightarrow x + z > y + z$
\item $x > y \land z >0 \rightarrow x * z > y * z$
\item   $\forall A \subseteq \mathbb{R} (A \neq \phi \land \exists x \in \mathbb{R} (\forall y \in A (y \leq x)) \rightarrow \exists z \in R ((\forall y \in A (y \leq z) )\land ( \forall x \in \mathbb{R} (\forall y \in A (y \leq x) \rightarrow z \leq x))))$
\end{enumerate}
\end{frame}

\end{CJK*}
\end{document}

%%% Local Variables:
%%% mode: latex
%%% TeX-master: t
%%% End:
