\documentclass{article}
\usepackage{tikz}
\usepackage{CJKutf8}
\usepackage{amsmath}
\usepackage{amsthm}
\begin{document}
\begin{CJK}{UTF8}{gbsn}
\newtheorem*{Exercise}{习题}
\begin{Exercise}设$G$为一个$(p,q)$图,证明:$\chi(G) \geq p^2/(p^2-2q)$
\end{Exercise}
\begin{proof}[证明]  设图$G$的色数$\chi(G)=n$,则$G$可以用$n$种颜色进行顶点着色使得相邻的顶点着不同的颜色。设着不同颜色的顶点的个数分别为$p_1,p_2,\ldots,p_n$,则
  \[p_1+p_2+\ldots +p_n = p_1 \cdot 1 + p_2\cdot 1 + \ldots + p_n\cdot 1 \leq \sqrt{p_1^2 + p_2^2 + \ldots + p_n^2} \cdot \sqrt{n}\]
  于是
  \[\sqrt{n} \geq \frac{p_1+p_2+\ldots + p_n}{\sqrt{p_1^2 + p_2^2 + \ldots + p_n^2}}\]
  从而
  \begin{equation}
    \label{eq:1}
  n \geq \frac{p^2}{p_1^2 + p_2^2 + \ldots + p_n^2}    
  \end{equation}
  再由
  \[(p_1+p_2+\ldots + p_n)^2 = p_1^2 + p_2^2 + \ldots + p_n^2 + 2\sum_{1\leq i<j\leq n}p_ip_j\]
  知
  \[p_1^2 + p_2^2 + \ldots + p_n^2=p^2-2\sum_{1\leq i<j\leq n}p_ip_j\leq p^2-2q\]
  结合\eqref{eq:1}式知
  \[n \geq p^2/(p^2-2q)\]
  结论得证。
\end{proof}

\end{CJK}
\end{document}


%%% Local Variables:
%%% mode: latex
%%% TeX-master: t
%%% End:
