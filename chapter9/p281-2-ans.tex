\documentclass{article}
\usepackage{tikz}
\usepackage{CJKutf8}
\usepackage{amsmath}
\usepackage{amsthm}
\begin{document}
\begin{CJK}{UTF8}{gbsn}
  \newtheorem*{Exercise}{习题}
\begin{Exercise}
  设$G$为一个有$k$个支的平面图。如果$G$的顶点数、边数、面数分别为$p$,$q$和$f$,试证:
  \[p-q+f=k+1\]
\end{Exercise}
\begin{proof}[证明]
  设$G$的的$k$个支分别为$G_1$,$G_2$,$\ldots$,$G_k$,其中$G_i$有$p_i$个顶点,$q_i$条边,$f_i$个面($1\leq i \leq k$)。

  由欧拉公式知,
  \begin{align*}
    p_1-q_1+f_1&=2\\
    p_2-q_2+f_2&=2\\
    \cdots&\\
    p_k-q_k+f_k&=2
  \end{align*}
  以上各式相加得:
  \[(p_1+p_2+\ldots+p_k)-(q_1+q_2+\ldots+q_k)+(f_1+f_2+\ldots+f_k)=2k\]

由$G$只有一个外部面知
\[f_1 + f_2 + \ldots + f_k = f + (k-1)\]
从而
\[p-q+f+(k-1)=2k\]
即
\[p-q+f = k + 1\]
\end{proof}

\end{CJK}
\end{document}


%%% Local Variables:
%%% mode: latex
%%% TeX-master: t
%%% End:
