\documentclass{article}
\usepackage{tikz}
\usepackage{CJKutf8}
\usepackage{amsmath}
\usepackage{amsthm}
\begin{document}
\begin{CJK}{UTF8}{gbsn}
\newtheorem*{Exercise}{习题}
\begin{Exercise}[p328-1]
用数学归纳法证明每个比赛图中必有有向哈密顿路。
\end{Exercise}
\begin{proof}[证明]
  用数学归纳法证明,施归纳于顶点数$p$。

  (1)当$p=1$时,结论显然成立。

  (2)假设当$p=k(k\geq 1)$时结论成立,往证当$p=k+1$时结论也成立。设$D=(V,A)$为一个包含$k+1$个顶点的比赛图,$v$为$D$的任意一个顶点,则$D-v$为一个包含$k$个顶点的比赛图。由归纳假设,$D-v$有一条有向哈密顿路$v_1v_2\ldots v_k$。如果$(v,v_1)\in A$,那么$vv_1v_2\ldots v_k$为有向图$D$的一条有向哈密顿路;如果$(v_k,v)\in A$,那么$v_1v_2\ldots v_kv$为有向图$D$的一条有向哈密顿路;如果$(v,v_1)\notin A$并且$(v_k,v)\notin A$,由$D$为比赛图知$(v_1,v)\in A$并且$(v,v_k)\in A$,在$1,2,\ldots, k$中选取最大的下标$i$使得$(v_i,v)\in A$,则$(v,v_{i+1})\in A$,于是$v_1,v_2,\ldots,v_i,v,v_{i+1},\ldots,v_k$为$D$的一条有向哈密顿路。 
\end{proof}

\end{CJK}
\end{document}


%%% Local Variables:
%%% mode: latex
%%% TeX-master: t
%%% End:
