\documentclass{article}
\usepackage{tikz}
\usepackage{CJKutf8}
\usepackage{amsmath}
\usepackage{amsthm}
\begin{document}
\begin{CJK}{UTF8}{gbsn}
\newtheorem*{Exercise}{习题}
\begin{Exercise}
  设$G$为一个三次正则图,试证:$\kappa(G)=\lambda(G)$
\end{Exercise}
\begin{proof}[证明]
  (1)如果$\kappa(G)=0$,则$G$不连通,此时$\lambda(G)=0$,故$\kappa(G) = \lambda(G)$。

  (2)如果$\kappa(G)=1$,则$G$中存在顶点$u$,$G-u$不连通。由$\deg u=3$知,$G-u$至少存在一个分支只有一条边与$u$相连,显然去掉这条边之后,$G$不连通,所以$\lambda(G)=1$,故$\kappa(G)=\lambda(G)$。

  (3)如果$\kappa(G)=2$,则存在两个顶点$v_1$和$v_2$,$G-\{v_1,v_2\}$不连通。$G-v_1$是连通的,且$G-v_1-v_2$不连通,类似于(2)中的讨论知$G-v_1$中存在一条边$e_2$,$G-v_1-e_2$不连通。另一方面由$\lambda(G)\geq \kappa(G)=2$知$G-e_2$是连通的,由于$G-e_2-v_1=G-v_1-e_2$不连通,由与(2)类似的讨论知$G-e_2$中存在一条边$e_1$,$G-e_2-e_1$不连通,所以$\lambda(G)=2$,故$\kappa(G)=\lambda(G)$。

  (4)如果$\kappa(G)\geq 3$,由$\kappa(G) \leq \lambda(G) \leq \delta(G) = 3$知,$\kappa(G)=\lambda(G)=3$。
\end{proof}

\end{CJK}
\end{document}


%%% Local Variables:
%%% mode: latex
%%% TeX-master: t
%%% End:
