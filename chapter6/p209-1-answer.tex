\begin{proof}[答]
  设$u$与$v$是图$G$的两个不同顶点。如果$u$与$v$间有两条不同的通道,则$G$中不一定有圈。举例如下:考虑$G=(\{u,v\},\{(u,v)\})$,则$uv$和$uvuv$为$u$与$v$间两条不同的通道,但$G$中没有圈。

  如果$u$与$v$间有两条不同的迹,则$G$中一定有圈。证明如下:设$u$与$v$间有两条不同的迹$T_1$和$T_2$。如果$T_1$和$T_2$都为路,则$G$中有圈;如果$T_1=uv_1v_2\ldots v_nv$不是路,设$v_j=v_i(i<j)$为第一个重复的顶点,则$v_iv_{i+1}\ldots v_j$构成$G$中的一个圈;同理,如果$T_2$不是路,$G$中有圈。
  
\end{proof}