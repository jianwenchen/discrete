\documentclass{article}
\usepackage{tikz}
\usepackage{CJKutf8}
\usepackage{amsmath}
\usepackage{amsthm}
\begin{document}
\begin{CJK}{UTF8}{gbsn}
\newtheorem*{Ex}{习题}
  \begin{Ex}
      证明:唯一没有三角形的$(p,[\frac{p^2}{4}])$图为$K(\lfloor \frac{p}{2} \rfloor,\lceil \frac{p}{2} \rceil )$。
\end{Ex}
\begin{proof}[证法一]用数学归纳法证明以下结论:唯一没有三角形的包含$p$个顶点且边数$q\geq [\frac{p^2}{4}]$的图一定为$K(\lfloor \frac{p}{2} \rfloor,\lceil \frac{p}{2} \rceil )$。施归纳于顶点数$p$,只证$p$为奇数的情况,$p$为偶数的情况是类似的。

  1) 当$p=1$时,唯一没有三角形的包含一个顶点且边数$q\geq 0$的图一定为$K(0,1)$,结论显然成立。(注:我们把$(1,0)$图也称为偶图,并记为$K(0,1)$或 $K(1,0)$)。

  

  2)假设当$p=2k-1(k\geq 1)$时结论成立,往证当$p=2k+1$时结论也成立。设$G$为一个没有三角形,顶点数$p=2k+1$,边数$q \geq [\frac{p^2}{4}]$的图。显然,$G$中至少有两个邻接的顶点$u$和$v$。图$G'=G-\{u\}-\{v\}$中没有三角形,有$2k-1$个顶点。因为$G$中没有三角形,如果$u$与$G'$的$x$个顶点邻接,则$v$至多能与$G'$中剩余的$2k-1-x$个顶点邻接,于是$G'$中的边数
  \begin{equation*}
    \begin{split}
      q'&\geq q - x - (2k-1-x) - 1\\
      &=[\frac{(2k+1)^2}{4}]-2k\\
      &=k^2-k\\
      &=[\frac{(2k-1)^2}{4}]
    \end{split}
  \end{equation*}

  由归纳假设,$G'$为$K(\lfloor \frac{2k-1}{2} \rfloor,\lceil \frac{2k-1}{2} \rceil )$,即$K(k-1,k)$。以下证明$G$必为$K(k,k+1)$。假设偶图$G'$的顶点集有一个二划分为$\{V_1,V_2\}$,使得$G'$的任意一条边的两个端点一个在$V_1$中,一个在$V_2$中,$|V_1|=k-1$,$|V_2|=k$。由$G$中没有三角形知$V_1$和$V_2$中的每个顶点在$G$中至多与顶点$u$和顶点$v$中的一个邻接。另外,$V_1$和$V_2$中的每个顶点在$G$中必与顶点$u$和顶点$v$中的一个邻接,否则,$G$中的边数$q < (k-1)k + (2k-1) + 1 = k^2 + k = [\frac{(2k+1)^2}{4}]$,矛盾。不妨设在$G$中$V_2$中的某个顶点与$v$相邻接,由$G$中没有三角形知$v$不能与$V_1$中的顶点相邻接,从而$u$与$V_1$中每个顶点相邻接,$u$与$V_2$中的每个顶点相邻接。这证明了$G$为$K(k,k+1)$。
  
  
\end{proof}

\begin{proof}[证法二]
  
\end{proof}
\end{CJK}
\end{document}


%%% Local Variables:
%%% mode: latex
%%% TeX-master: t
%%% End:
